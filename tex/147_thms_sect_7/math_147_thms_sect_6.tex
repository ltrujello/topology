\documentclass[a4paper,12pt,twoside]{hmcpset}
\usepackage[utf8]{inputenc}
\usepackage[english]{babel}
\usepackage{fancyhdr}
\usepackage[margin=.6in]{geometry}
\usepackage{graphicx}
\usepackage{amsmath}
\usepackage{mathtools}
\usepackage[mathscr]{euscript}
%\newcommand*{\ms}[1]{\ensuremath{\mathscr{#1}}}

\pagestyle{fancy}
\fancyhf{}
\rhead{Spring 2019}
\lhead{\vspace{5mm} Math 147 Topology}
\rfoot{Page \thepage}
 
\renewcommand{\headrulewidth}{2pt}
\renewcommand{\footrulewidth}{1pt}

\graphicspath{ {./figures_theorems/} } 

% info for header block in upper right hand corner
\begin{document}
\section*{Chapter 6\\ Countable Features of Spaces: Size Restrictions}

\textbf{Exercise 6.1} Show that $A$ is dense in $X$ if and ony if every 
non-empty open set of $X$ contains a point of $A$.
\\
\\
First we prove the forward direction. Suppose that $A$ is a dense subset in 
$X$. Then by definition, $\overline{A} = X$. Thus every point of $X$ 
is a limit point of $A$, which means that for every point $p \in X$ and 
every open set $U$ which contains $p$ we see that 
$$
(U - \{p\})\cap A \ne \emptyset.
$$
Since this holds for all $p \in X$, we see that every open set in $X$ must 
contain points in $A$, which proves this direction.
\\
\\
Now suppose that every nonempty open set of $X$ contains a point of $A$. 
Then this means that for any $p \in X$, any open set $U$ containing $p$ 
must also contain a point in $A$. By definition, this is a limit point.
Since $p$ was an arbitrary point of $x$, we must have that every element 
of $X$ is a limit point of $A$. Therefore, we must have that 
$\overline{A} = X$, which finishes the proof in this direction. 
\\
\\
\textbf{Exercise 6.2} Show that $\mathbb{R}_{\text{std}}$ is separable.
With which of the topologies on $\mathbb{R}$ that you have studied is
$\mathbb{R}$ not separable?
\\
\\
Observe that a countable dense subset in $\mathbb{R}_{\text{std}}$ is the 
set of rationals. This is because every nonempty open set of $\mathbb{R}$
on the standard topology contains points of $\mathbb{Q}$. By our previous 
exercise, this allows us to conclude that $\mathbb{Q}$ is dense in $\mathbb{R}$.
Since the rationals are countable, this in total allows us to conclude that 
$\mathbb{R}_{\text{std}}$ has a countable dense subset, and is therefore 
separable by definition. \\
However, this wouldn't hold for the discrete 
topology on $\mathbb{R}$, since it does not have a countable dense subset 
with this topology. The countable complement is also not separable, since 
every open set in the topology must be uncountable and hence finding a 
countable but dense subset of $X$ is impossible.
\\
\\
\textbf{Exercise 6.4} Find a separable space that contains a subspace that 
is not separable in the subspace topology.



\begin{problem}[Theorem 6.5]
    If $X$ and $Y$ are separable spaces, then $X \times Y$ is separable.
\end{problem}

\begin{solution}
    Suppose $X$ or $Y$ are separable spaces. Then there exist countable sets 
    $A$ and $B$ such that $\overline{A} = X$ and $\overline{B} = Y$. 
    Using the fact that $\overline{A} \times \overline{B} = \overline{A \times B}$,
    we see that
    $$
    \overline{A \times B} = \overline{A} \times \overline{B} = X \times Y.
    $$
    Thus $A \times B$ is dense in $X \times Y$. But also observe that 
    $A \times B$ is countable, since we can form a bijection between $A 
    \times B$ and $A$ or $B$ (namely the projection function). Thus $X \times Y$
    must be separable because it contains a countable dense subset, which is 
    what we set out to show.
\end{solution}

\begin{problem}[Theorem 6.6]
    The space $2^{\mathbb{R}}$ is separable.
\end{problem}

\begin{solution}
        
\end{solution}

\begin{problem}[Theorem 6.9]
    Let $X$ be a $2^{\text{nd}}$ countable space. Then $X$ is separable.
\end{problem}

\begin{solution}
    Let $p_i$ be some point of $B_i$, $i \in \mathbb{N}$, where $B_i$ is a basic open set 
    from our countable basis. Then for any open set $V$ of $X$, 
    we know that $V$ will intersect $\{p_i\}_{i \in \mathbb{N}}$ 
    since by definition $V$ must contain some basic open set $B_i$ for 
    which $p_i \in B_i$. Thus by Exercise 6.1, $\{p_i\}$ is dense, and 
    since it's countable we have that $X$ is separable.
\end{solution}

\begin{problem}[Theorem 6.11]
    Every uncountable set in a $2^{\text{nd}}$ countable space has a limit point. 
\end{problem}

\begin{solution}
    Suppose we have an uncountable set $A$ in $X$, and for the sake of
    contradiction suppose that $U$ has no limit points. Then every point
    of $A$ is an isolated point, which means that there exists an open set 
    $U$ such that $U \cap A = \{p\}$ for all $p \in A$. Note that
    for every such $U$ there exists a $B$ basic open set such that $B \subset U$.
    Thus $p \in B \subset U$. 
    However, there are only countably many basic open sets, while an uncountable 
    number of $p \in A$, which is a contradiction since we cannot contain an uncountable 
    number of points with a countable number of basic open sets. Thus $A$ 
    must have a limit point, which is what we set out to show. 
\end{solution}


\end{document}
