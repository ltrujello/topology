\documentclass[a4paper,12pt,twoside]{hmcpset}
\usepackage[utf8]{inputenc}
\usepackage[english]{babel}
\usepackage{fancyhdr}
\usepackage[margin=1in]{geometry}
\usepackage{graphicx}
\usepackage{amsmath}
\usepackage{mathtools}
\usepackage[mathscr]{euscript}
\usepackage{lmodern} % math, rm, ss, tt
\usepackage[T1]{fontenc}
\usepackage{relsize}
%\newcommand*{\ms}[1]{\ensuremath{\mathscr{#1}}}
\renewcommand{\labelenumi}{{\bf (\alph{enumi})}}

\pagestyle{fancy}
\fancyhf{}
\rhead{Spring 2019}
\lhead{\vspace{5mm} Math 147 Topology}
\rfoot{Page \thepage}
\chead{Section 9}
 
\renewcommand{\headrulewidth}{2pt}
\renewcommand{\footrulewidth}{2pt}

\graphicspath{ {./figures_theorems/} } 

% info for header block in upper right hand corner
\begin{document}
\section*{Chapter 9\\ Connectedness: When Things Don't Fall Into Pieces}

\begin{problem}[Theorem 9.1]
    The following are equivalent:
    \begin{itemize}
        \item[1.] $X$ is connected 
        \item[2.] there is no continuous function $f: X \rightarrow
        \mathbb{R}_{\text{std}}$ such that $f(X) = \{0, 1\}$
        \item[3.] $X$ is not the union of two disjoint nonempty
        separated sets 
        \item[4.] $X$ is not the union of two disjoint nonempty
        closed sets
        \item[5.] the only subsets of $X$ that are both closed and
        open in $X$ are both the empty set and $X$ itself
        \item[6.] for every pair of points $p$ and $q$ and every open
        cover $\{U_{\alpha}\}_{\alpha \in \lambda}$ of $X$ there
        exists a finite number of $U_\alpha$'s, $\{U_{\alpha_1},
        U_{\alpha_2}, \dots, U_{\alpha_n}\}$ such that $p \in
        U_{\alpha_1}$, $q \in U_{\alpha_n}$ for each $i < n$, 
        $U_{\alpha_i} \cap U_{\alpha_{i + 1}} \ne \emptyset$.   
    \end{itemize}
\end{problem}

\begin{proof}
    \begin{enumerate}
        \item[] ($\mathbf{1 \implies 2}$) Suppose $X$ is connected,
        and for contradiction that there is a continuous function $f:
        X \rightarrow \mathbb{R}_{\text{std}}$ such that $f(X) = \{0,
        1\}$. However, this would imply that $f^{-1}(1)$ and
        $f^{-1}(0)$ are (1) disjoint open sets in $X$ such that (2)
        their union is $X$. However, that contradicts the fact that
        $X$ is connected by definition. Therefore, there is no
        continuous function $f:
        X \rightarrow \mathbb{R}_{\text{std}}$ such that $f(X) = \{0,
        1\}$.
        \\
        \\
        ($\mathbf{2 \implies 1}$) Now if there is no continuous
        function $f:
        X \rightarrow \mathbb{R}_{\text{std}}$ such that $f(X) = \{0,
        1\}$, then that means $X$ cannot be split into two disjoint
        open sets whos union is $X$, which implies that $X$ is
        connected.
        \\
        \\
        ($\mathbf{1 \implies 3}$) Since $X$ is connected, it is not
        the union of two nonempty disjoint open subsets of $X$.
        However, suppose $A, B$ are two separated sets such that $A
        \cup B = X$.
        \\
        \\
        ($\mathbf{3 \implies 1}$) Suppose now that $X$ is not the
        union of two disjoint nonempty separated sets. Then $X$ is not
        union of two disjoint open sets, so that $X$ is connected. 
        \\
        \\
        ($\mathbf{1 \implies 4}$) Suppose $X$ is connected, and for
        contradiction that $X = A \cup B$ where $A$ and $B$ are
        disjoint nonempty closed
        sets. Then we can construct a continuous function from $f: X
        \to \{0, 1\}$, where $f^{-1}(0) = A$ and $f^{-1}(1) = B$.
        However, this contradictions the fact that $X$ is connected,
        so that $X$ is no the union of two disjoint nonempty closed
        sets. 
        \\
        \\
        ($\mathbf{4 \implies 1}$) Suppose $X$ is not the union of two
        disjoint nonempty closed sets. Then there is no continuous
        function $f: X \to \{0, 1\}$ since $f^{-1}(0)$ and $f^{-1}(1)$
        cannot be open or closed. Thus $X$ must be connected.
        \\
        \\
        ($\mathbf{1 \implies 5}$) Suppose $X$ is connected. Suppose
        there is a set such that $A \ne X$ and $A \ne \emptyset$ 
        is open and closed. Then $A^c
        \cup A = X$. However, that would mean $X$ is the union of two
        disjoint non empty open sets, which is a contradiction. Thus
        the only open and closed sets are $X$ and $\emptyset$.
        \\
        \\
        ($\mathbf{5 \implies 1}$) Suppose the only open and closed
        sets in $X$ are $X$ and $\emptyset.$ Suppose for contradiction
        that $X$ is not connected, so that $X = A \cup B$ for two
        disjoint nonempty open sets. Then $X^c = (A \cup B)^c = A^c
        \cap B^c = \emptyset$. However, this is a contradiction since
        their intersection must be nonempty. Therefore, $X$ is
        connected. 

        
    \end{enumerate}
\end{proof}

\begin{exercise}[Exercise 9.2]
\textbf{Exercise 9.2}
Which of the following spaces are connected?
\begin{enumerate}
    \item[1.] $\mathbb{R}$ with the discrete topology?
    \item[2.] $\mathbb{R}$ with the indiscrete topology?
    \item[3.] $\mathbb{R}$ with the finite complement topology?
    \item[4.] $\mathbb{R_\text{LL}}$?
    \item[5.] $\mathbb{Q}$ as a subspace of $\mathbb{R}_{\text{std}}$?
    \item[6.] $\mathbb{R} - \mathbb{Q}$ as a subspace of $\mathbb{R}_{\text{std}}$?      
\end{enumerate}
\end{exercise}

\begin{solution}
    \begin{enumerate}
        \item[1.] Every subset of $\mathbb{R}$ is
        open and closed. This violates Theorem 9.1(5) so that $\mathbb{R}$
        is not connected under the discrete topology.
    
        \item[2.] The only sets which are open and closed are
        $\mathbb{R}$ and $\emptyset$. Thus by Theorem 9.1(5) $\mathbb{R}$
        is connected under the indiscrete topology.
    
        \item[3.] For contradiction suppose there is 
        a set $U \subset \mathbb{R}$ which is open and closed
        and not $\mathbb{R}$ or the emptyset.  
    
        Since $U$ is open, $U^c$ is finite. However $(U^c)^c = U$ is
        infinite and hence $U^c$ is not an open set. But this contradicts
        the assumption that $U$ was open and closed. Thus $\mathbb{R}$ is
        connected on the finite complement topology. 
    
        \item[4.] Consider a basic open set $[a, b)$. Observe that 
        \[
            [a, b)^c = (-\infty, a) \cup [b, \infty)
        \]  
        which is the union of two open sets, and hence is open. Thus $[a,
        b)$ is open and closed. By Theorem 9.1.5, we have that
        $\mathbb{R}_{\text{LL}}$ is not connected.
    
        \item[5.] Observe that $(\mathbb{Q} \cap (-\infty, \pi))$ and
        $(\mathbb{Q} \cap (\pi, \infty))$ are disjoint, separated sets in
        the subspace $\mathbb{Q}$ and 
        \[
            (\mathbb{Q} \cap (-\infty, \pi)) \cap (\mathbb{Q} \cap (\pi, \infty))
            = \mathbb{Q}.
        \] 
        Thus $\mathbb{Q}$ is not open as a subspace of $\mathbb{R}_{\text{std}}$.
        
        \item[6.] Observe that $(\mathbb{R} - \mathbb{Q}) \cap (-\infty,
        0)$ and $(\mathbb{R} - \mathbb{Q}) \cap (0, \infty)$ are disjoint
        separated sets and 
        \[
            ((\mathbb{R} - \mathbb{Q}) \cap (-\infty, 0)) \cup 
            ((\mathbb{R} - \mathbb{Q}) \cap (0, \infty)) = \mathbb{R} - \mathbb{Q}.
        \]   
        Thus $\mathbb{R} - \mathbb{Q}$ is not connected.
    \end{enumerate}
\end{solution}

\begin{problem}[Theorem 9.3]
    The space $\mathbb{R}_{\text{std}}$ is connected.
\end{problem}

\begin{proof}
    The only closed and open sets in $\mathbb{R}_{\text{std}}$ are the
    emptyset and $\mathbb{R}$ itself, so that by Theorem 9.1(5) we
    can conclude that $\mathbb{R}_\text{std}$ is connected. 
\end{proof}


\begin{problem}[Theorem 9.4]
    Let $A$ and $B$ be separated subsets of a space $X$. If $C$ is a
    connected subset of $A \cup B$, then either $C \subset A$ or $C
    \subset B$.
\end{problem}

\begin{proof}
    Observe that if $C$ is a connected subset of $A \cup B$, where $A$
    and $B$ are separated in $X$, then $C$ is not the union of two
    disjoint open sets in the $A \cup B$ subspace topology. 
    \\
    \\
    Suppose for the sake of contradiction that $C \subset A$ and $C
    \subset B$. Then observe that 
    \[
       C \subset A \cap B = \emptyset   
    \]
    which is a contradiction since $C$ is nonempty. Thus it must be
    that $C \subset A$ or $C \subset B$.

\end{proof}

\begin{problem}[Theorem 9.5]
    Let $\{C_{\alpha}\}_{\alpha \in \lambda}$ be a collection of
    connected subsets of $X$ and $E$ another connected subset of $X$
    that for each $\alpha \in \lambda$, $E \cap C_{\alpha} \ne
    \emptyset$. Then $\displaystyle E \cup (\bigcup\limits_{\alpha \in \lambda}
    C_{\alpha})$ is connected.
\end{problem}

\begin{proof}
    Suppose for the sake of contradicition that $E \cup
    (\bigcup\limits_{\alpha \in \lambda}C_\alpha)$ is not connected. 
    Then $E \cup (\bigcup\limits_{\alpha \in \lambda}C_\alpha) = A \cup B$
    where $A$ and $B$ are some separated sets in $X$. Observe that
    since $E$ is a connected subset of $X$, we have by Theorem 9.4
    that $E \subset A$ or $E \subset B$. Without loss of generality
    suppose $E \subset A$. Then since each $C_\alpha$ is a connected
    subset of $A \cup B$, Theorem 9.4 implies that 
    $C_\alpha \subset B$ for at least one
    $\alpha \in \lambda$. However, this is a contradiction since $E
    \cap C_\alpha \ne \emptyset$ for all $\alpha \in \lambda$, while
    $A \cap B = \emptyset.$ Therefore, we must have that 
    $E \cup (\bigcup\limits_{\alpha \in \lambda})$ is connected. 
\end{proof}

\begin{problem}[Theorem 9.6]
    Let $C$ be a connected subset of the topological space $X$. If $D$
    is a subset of $X$ such that $C \subset D \subset \overline{C}$,
    then $D$ is connected.
\end{problem}

\begin{proof}
    Suppose that $C$ is a connected subset of $X$ and for the sake of
    contradiction that $D$ such that $C \subset D \subset 
    \overline{C}$ is not connected. Then there exists disjoint open
    sets $A$ and $B$ such that $A \cup B = D$. 
    Since $C$ is connected, we know by Theorem 9.5 that $C \cap A  =
    \emptyset$ or $C \cap B = \emptyset$. Without loss of generality,
    suppose that $C \cap A = \emptyset$. Then this is a contradiction
    since $A \subset D \subset \overline{C}$. Therefore, we must have
    that $D$ is connected. 
\end{proof}  

\begin{problem}[Theorem 9.8]
    Let $X$ be a topological space, $C$ a connected subset of $X$, and
    $X - C = A\big| B$. Then $A \cup C$ and $B \cup C$ are each
    connected 
\end{problem}

\begin{proof}
    Suppose that $X - C = A \cup B$ where $A$ and $B$ are separated.
    Now suppose that $A \cup C$ is not connected, so that 
    $A \cap C = U \cup V$ where $U, V$ are open. Now suppose that 
    $U \cap C \ne \emptyset$ and $V \cap C \ne \emptyset$. Then $(U
    \cap C) \cup (V \cap C) = A \cap C$ 
\end{proof}

\noindent
\textbf{Presented in class 4/3/19}\\
\begin{problem}[Theorem 9.12]
    Let $f: X \to Y$ be a continuous, surjective function. If $X$ is
    connected, then $Y$ is connected. 
\end{problem}

\begin{proof}
    Suppose $f: X \to Y$ is a continuous, surjective function. We can
    do proof by contradiction. Suppose $X$ is connected but
    $Y$ is not connected. By Theorem 9.1 part 5, there exists a set $V
    \subset Y$, $V \ne \emptyset$ $V \ne Y$, such
    that $V$ is open and closed in $Y$. By continuity, $f^{-1}(V)$
    is both open and closed in $X$, and by surjectivity, $f^{-1}(V)$
    is a proper subset of $X$. Thus $X$ has an open and closed set,
    one which is not $\emptyset$ or $X$, which contradicts the fact that
    $X$ is not connected by
    Theorem 9.1 part 5. Thus if $X$ is connected, $Y$
    is connected, as desired.
\end{proof}

\begin{problem}[Theorem 9.13] (Intermediate Value Theorem!)
    Let $f: \mathbb{R}_{\text{std}} \to \mathbb{R}_{\text{std}}$ be a
    continuous map. If $a, b \in \mathbb{R}$ and $r$ is a point of
    $\mathbb{R}$ such that $f(a) < r < f(b)$ then there exists a point
    $c$ in $(a, b)$ such that $f(c) = r$
    
\end{problem}

\begin{proof}
    Observe that $\mathbb{R}_{\text{std}}$ is connected. Since $f:
    \mathbb{R}_{\text{std}} \to \mathbb{R}_{\text{std}}$, connected
    should be preserved.
    \\
    \\
    Suppose there does not exist a point $c \in (a, b)$ such
    that $f(c) = r$. Then $f(x) < r$ or $r < f(x)$ for all $x \in (a,
    b)$. However since $f(\mathbb{R}) = \mathbb{R}$, 
    this implies that $\mathbb{R}_{\text{std}}$ is not
    connected, which contradicts the fact that $\mathbb{R}_\text{std}$
    is connected. Therefore such a $c$ must exist.  
\end{proof}

\begin{problem}[Theorem 9.18]
    Each component of $X$ is connected, closed, and not contained in
    any strictly larger connected subset of $X$.
\end{problem}

\begin{proof}
    Consider a component $C = \bigcup\limits_{\alpha \in \lambda}
    C_\alpha$ of $p$ in $X$, where each $C_\alpha$ is connected and $p \in
    C_\alpha$ for all $\alpha \in \lambda$.
    Observe that we can apply Theorem 9.5 to conclude that $C$ is
    connected, since (1) no member of the union of $C$ is disjoint
    from any other member (as they all contain $p$) and (2) each
    member is connected. 
    \\
    \\
    Suppose that $C$ is not closed. Then there is a point $q \not\in C$
    and an open set $U$ containing $q$ such that $(U - \{q\}) \cap C
    \ne \emptyset$. 


\end{proof}

\begin{problem}[Theorem 9.35]
    A path connected space is connected.
\end{problem}

\begin{proof}
    Suppose $X$ is path connected but not connected. Then there exist
    two disjoint open subsets $A ,B$ such that $A \cup B = X$. Observe
    that any point in $A$ cannot be joined together with any point $B$
    by a path, a contradiction to the path connectivity of $X$. Thus
    $X$ must be connected. 
\end{proof}

\begin{problem}[Theorem 9.36]
    The flea and comb space is connected but not pathwise connected.
    (The flea and comb space is the union of the topologist's comb and
    the point (0, 1).)
\end{problem}

\begin{proof}
    Let $A$ be the set of the comb space. This is obviously path
    connected, and so it is connected by Theorem 9.35. Observe now
    that 
    \[
        A \subset A \cup {flea} \subset \overline{A}
    \]
    so that $A \cup {flea}$, the flea and comb space, must be
    connected. 
\end{proof}




\end{document}