\documentclass[a4paper,12pt,twoside]{hmcpset}
\usepackage[utf8]{inputenc}
\usepackage[english]{babel}
\usepackage{fancyhdr}
\usepackage[margin=.6in]{geometry}
\usepackage{graphicx}
\usepackage{amsmath}
\usepackage{mathtools}
\usepackage[mathscr]{euscript}
\usepackage{lmodern} % math, rm, ss, tt
\usepackage[T1]{fontenc}
\usepackage{relsize}
%\newcommand*{\ms}[1]{\ensuremath{\mathscr{#1}}}
\renewcommand{\labelenumi}{{\bf (\alph{enumi})}}

\pagestyle{fancy}
\fancyhf{}
\rhead{Spring 2019}
\lhead{\vspace{5mm} Math 147 Topology}
\rfoot{Page \thepage}
\chead{Section 10}
 
\renewcommand{\headrulewidth}{2pt}
\renewcommand{\footrulewidth}{2pt}

\graphicspath{ {./figures_theorems/} } 

% info for header block in upper right hand corner
\begin{document}
\section*{Chapter 10\\ Metric Spaces: Getting some distance}

\noindent
\textbf{4/8/19 Q: Why does X have to be a metric space?}\\
\begin{problem}[Lebesgue Number Theorem 10.24] 
    Let $\{U_{\alpha}\}_{\alpha \in \lambda}$ be an open cover of a
    compact set $A$ in a metric space $X$. Then there exists a
    $\delta > 0$ such that for every point $p \in A$, $B(p, \delta)
    \subset U_\alpha$ for some $\alpha$. This number $\delta$ is called a \textbf{Lebesgue
    number} of the cover.
\end{problem}

\begin{proof} 
    Since $A$ is compact, there exists a finite subcover
    $\{U_{\alpha_1}, U_{\alpha_2}, \dots, U_{\alpha_n}\}$.
    Now suppose for the sake of contradiction that there does not exists
    such a  $\delta$.
    Then in order for this to happen, we would need
    that for every $B(p, \delta)$ containing $p$ there exists a member
    of $U_{\alpha'} \in \{U_{\alpha_1}, U_{\alpha_2}, \dots,
    U_{\alpha_n}\}$ such that 
    $B(p, \delta) \not\subset U_{\alpha'}$. But since we can
    theoretically propose an infinite number of $\delta$, we must have
    an infinite number of such $U_\alpha'$s. 
    \\
    \\
    However, we cannot do
    this in the finite subcover, as it is finite. Therefore the
    contrary must be true: there exists a $\delta$ such that for every
    $p \in A$, $B(p, \delta) \subset U_\alpha$ for some $\alpha$. And
    since this is true for the finite subcover, which is a subset of
    the open cover, this is definitely true for the open cover.

\end{proof}


\begin{problem}[Theorem 10.25]
    Let $\gamma : [0, 1] \to X$ be a \textbf{path}: a continuous map
    from $[0, 1]$ into the space $X$. Given an open cover
    $\{U_\alpha\}$ of $X$, show that $[0, 1]$ can be divided into $N$
    intervals of the form $I_i = [\frac{i - 1}{N}, \frac{i}{N}]$ 
    such that
    each $\gamma(I_i)$ lies completely in one set of the cover.    
\end{problem}

\begin{proof}
    If $\{U_\alpha\}_{\alpha \in \lambda}$ is an open cover of $X$,
    then consider the set $\{\gamma^{-1}(U_\alpha)\}_{\alpha \in
    \lambda}$. This will be an open cover of $\gamma$, since we know
    $\gamma$ maps $[0, 1]$ into $X$. However, since $\gamma$ is
    compact, we know by Lebesgue Number Theorem that there exists a
    $\delta$ such that $p \in B(p, \delta) \subset
    \gamma^{-1}(U_\alpha)$ for all $p \in [0, 1]$ where
    $\gamma^{-1}(U_\alpha)$ is some set in the open cover containing
    $p$.
    \\
    \\
    Let $\dfrac{1}{N} < \delta$ where $N$ is a positive integer. Then
    observe that the sequence of intervals 
    \[
        \left[\frac{i-1}{N}, \frac{i}{N}\right] \quad 1 \le i \le N  
    \]
    will each be contained in at least one member of
    $\gamma^{-1}(U_\alpha)$. Thus 
    \[
        \left[\frac{i-1}{N}, \frac{i}{N}\right] \subset \gamma^{-1}(U_\alpha) 
        \implies \gamma\left(\left[\frac{i-1}{N}, \frac{i}{N}\right]\right) \subset U_\alpha.  
    \] 
    Thus $[0, 1]$ can be divided into  $N$
    intervals of the form $I_i = [\frac{i - 1}{N}, \frac{i}{N}]$ 
    such that
    each $\gamma(I_i)$ lies completely in one set of the cover in $X$,
    which is what we set out to show.
\end{proof}


\end{document}